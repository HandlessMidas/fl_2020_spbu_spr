\documentclass[12pt]{article}
\usepackage[left=2cm,right=2cm,top=2cm,bottom=2cm,bindingoffset=0cm]{geometry}
\usepackage[utf8x]{inputenc}
\usepackage[english,russian]{babel}
\usepackage{cmap}
\usepackage{amssymb}
\usepackage{amsmath}
\usepackage{url}
\usepackage{pifont}
\usepackage{tikz}
\usepackage{verbatim}
\usepackage[most]{tcolorbox}

\usetikzlibrary{shapes,arrows}
\usetikzlibrary{positioning,automata}
\tikzset{every state/.style={minimum size=0.2cm},
initial text={}
}

\newenvironment{myauto}[1][3]
{
  \begin{center}
    \begin{tikzpicture}[> = stealth,node distance=#1cm, on grid, very thick]
}
{
    \end{tikzpicture}
  \end{center}
}


\begin{document}
\begin{center} {\LARGE HW05} \end{center}

\bigskip

\begin{enumerate}
\item[2.]
$S \to RS|R$\\
$R \to aSb|cRd|ab|cd|\epsilon$\\\\
Добавим стартовый нетерминал:\\
$S_{start} \to S$\\
$S \to RS|R$\\
$R \to aSb|cRd|ab|cd|\epsilon$\\\\
Избавимся от неодиночных терминалов:\\
$S_{start} \to S$\\
$S \to RS|R$\\
$R \to ASB|CRD|AB|CD|\epsilon$\\
$A \to a$\\
$B \to b$\\
$C \to c$\\
$D \to d$\\\\
Устраним длинные правила:\\
$S_{start} \to S$\\
$S \to RS|R$\\
$R \to AQ|CP|AB|CD|\epsilon$\\
$A \to a$\\
$B \to b$\\
$C \to c$\\
$D \to d$\\
$Q \to SB$\\
$P \to RD$\\\\
Устраним $\epsilon$-правила:\\
$S_{start} \to S|\epsilon$\\
$S \to RS|R|S$\\
$R \to AQ|CP|AB|CD$\\
$A \to a$\\
$B \to b$\\
$C \to c$\\
$D \to d$\\
$Q \to SB|B$\\
$P \to RD|D$\\\\
Устраним цепные правила:\\
$S_{start}\to RS|AQ|CP|AB|CD|\epsilon$\\
$S \to RS|AQ|CP|AB|CD$\\
$R \to AQ|CP|AB|CD$\\
$A \to a$\\
$B \to b$\\
$C \to c$\\
$D \to d$\\
$Q \to SB|b$\\
$P \to RD|d$\\

Done.
\item[3.] КС грамматика для языка:\\
$S \to aaS|aSb|Sbb|ab|aa|bb$\\
Поймем, что у нас получаются все слова вида $a^nb^m$, $n + m > 0$, $(n + m) \, \vdots \, 2$ и только они.\\
Сначала то, что все. Рассмотрим случаи:\\
1) $m = 0 \Rightarrow n>0$ - чётное.\\
Применим $\frac{n}{2} - 1$ раз $S \to Sbb$ и один раз $S \to bb$. Получим $S \to Sbb \to \dots \to Sbb \dots bb \to b^n$ (если n = 0, то то же самое)\\
2) $m, n > 0$, чётные.\\
Применим $\frac{n}{2}$ раз $S \to Sbb$,  $\frac{m}{2}-1$ раз $S \to aaS$, и один раз $S \to aa$\\
3) $m, n > 0$, нечётные.\\
Применим $\frac{n-1}{2}$ раз  $S \to Sbb$,  $\frac{m-1}{2}$ раз $S \to aaS$, и один раз $S \to ab$\\\\
Теперь докажем, что никакие другие не получатся. Понятно, что нет пустого слова. Понятно (например по индукции), что все буквы a левее букв b. А так как в каждом правиле четность суммы сохраняется, то и в конце она тоже четная.
 \end{enumerate}
\end{document}