\documentclass[12pt]{article}
\usepackage[left=2cm,right=2cm,top=2cm,bottom=2cm,bindingoffset=0cm]{geometry}
\usepackage[utf8x]{inputenc}
\usepackage[english,russian]{babel}
\usepackage{cmap}
\usepackage{amssymb}
\usepackage{amsmath}
\usepackage{url}
\usepackage{pifont}
\usepackage{tikz}
\usepackage{verbatim}
\usepackage[most]{tcolorbox}

\usetikzlibrary{shapes,arrows}
\usetikzlibrary{positioning,automata}
\tikzset{every state/.style={minimum size=0.2cm},
initial text={}
}

\newenvironment{myauto}[1][3]
{
  \begin{center}
    \begin{tikzpicture}[> = stealth,node distance=#1cm, on grid, very thick]
}
{
    \end{tikzpicture}
  \end{center}
}


\begin{document}
\begin{center} {\LARGE Домашнее задание на зачет} \end{center}
\begin{center} {Аргунов Данил, Б09} \end{center}

\bigskip
	
\begin{enumerate}
	\item $S \rightarrow aS | aSb | ab$\\
	Почему получаются все слова? Допустим есть слово $a^n b^m$. Применим правило $S \rightarrow aS$ $n-m$ раз, потом $m-1$ раз правило $S \rightarrow aSb$ и наконец один раз $S \rightarrow ab$.\\
	Почему только они? Очевидно, что буквы $a$ всегда левве букв $b$ и их точно больше либо равно. Третье правило нужно, чтобы $n, m \geqslant 1$.
	\item 
	
\end{enumerate}
\end{document} 